% !TEX root = ../thesis.tex
%
\chapter{A Large Real Noisy Image Dataset, with A Comprehensive Evaluation of State-of-the-arts}
\label{sec:dataset}

\section{Introduction}

The statistical peorperties of realistic noise has been thoroughly studied for CCD and CMOS image sensors. The realistic noise related to the CCD or CMOS sensors could be divided into five major sources, including photon shot noise, fixed pattern noise, dark current, readout noise, and quantization noise, etc.

The shot noise is one inevitable source of noise and induced by the stochastic arrival process of photons to the sensor. This can be modeled by Possion distribution. This type of noise is proportional to the mean intensity of the specific pixel and is not stationary across the whole image. The fixed pattern noise include pixel response non-uniformity (PRNU) noise and dark current non-uniformity (DCNU) noise. PRNU means for a fixed light level, each pixel will have a slightly different output levels or responses. The major reason for the PRNU noise is the loss of light and color mixture in the neighboring pixels.

The dark current and fixed pattern noise are from the electronics within the sensor chip, while the readout noise and quantization noise are from discretization. The dark current is generated mainly  due to thermal agitation, even thouth there is no light reaching the camera sensor. The readout noise is mainly generated during the process of Charge to voltage conversion. The readout is inheretantly not accurate. This is also the reason for quantization noise, in which the readout value is quantizated to be an integer. The final pixel values are only discretization of the original raw pixel values. Other types of noise include CCD specific sources such as transfer efficiency and CMOS specific sources such as column noise.

As we have presented in the previous chapters, the image denoising task is very differetn when dealing with real noisy images when compared to the synthetic additive white Gaussian noise (AWGN). This is due to that the realistic noise is signal dependent while the AWGN noise is independent of the signals. Another key reason is that in-camera imaging pipeline will make the noise much more complex when compared with that in the raw image data.

Another issue about the real noisy image denoising is that the images with realistic noise has no ideal ground truth, i.e., the noise-free image. Hence, the objective evaluation about the image quality on the images denoised by existing methods should be nearly impossible. However, the image quality assessment by subjectiva evaluation would be very biased due to the limited number of subjectives in the evaluation experiments and the control of the experimental process. The missing of ground truth will be very problematic we have no objective measurements to evaluate the image quality of the images denoised by different methods. 

In order to address the challenge of missing ground truth of the realistic noisy images, we propose to construct a dataset for realistic photographs with reasonble noise-free ground truth images. Our basic idea is inspired by the paper of \cite{crosschannel}, in which a small dataset is set up for analysising the properties of realistic noise.


\section{Related Work}





\section{Dataset Construction}


\section{Analysis and Discussion}


\section{Experiments}










\section{Conclusion}