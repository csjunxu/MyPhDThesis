% !TEX root = ../thesis.tex
%
\chapter{Literature Review}
\label{sec:review}

%\cleanchapterquote{"Mens cujusque is est Quisque" – "Mind Makes the Man"}{Samuel Pepys}{}

In this chapter, I will review the denoising literature during the past decades. Since the additive white Gaussian noise is approximately similar to the read our noise in the real images, we first review the major methods existed in this area. Though these methods are proposed to deal with the AWGN noise, the idea can be applied to the other image denoising tasks such as real color image denoising. In the second part, I will review the existing methods proposed for processing real noisy images. Though the methods in this domain is not that versartile than those methods for the AWGN noise, the real noisy image denoising is the current mainstream for the image denoising community. Due to the noise is not known beforehand, noise estimation should be performed for the real noisy image denoising task. Hence, we also review the literature on image noise estimation task. 


\section{Synthetic Grayscale Image Denoising}
\label{sec:review:sys}

As a classical problem in low level vision, image denoising has been extensively studied, yet it is still an active topic for that it provides an ideal test bed for image modeling techniques. In general, image denoising aims to recover the clean image $\mathbf{x}$ from its noisy observation $\mathbf{y} = \mathbf{x} + \mathbf{v}$, where $\mathbf{v}$ is assumed to be additive white Gaussian noise.\ A variety of image denoising methods have been developed in past decades, including filtering based methods \cite{Tomasi1998}, diffusion based methods \cite{PeronaMalik1990}, total variation based methods \cite{rudin1992nonlinear,osher2005iterative}, wavelet/curvelet based methods \cite{softthresholding,bayesshrink,curvelet}, sparse representation based methods \cite{ksvd,lssc,ncsr}, nonlocal self-similarity based methods \cite{nlm,bm3d,nnm,wnnm}, etc.

Image modeling plays a central role in image denoising. By modeling the wavelet transform coefficients as Laplacian distributions, many wavelet shrinkage based denoising methods such as the classical soft-thresholding \cite{softthresholding} have been proposed.\ Chang et al. modeled the wavelet transform coefficients as generalized Gaussian distribution, and proposed the BayesShrink \cite{bayesshrink} algorithm.\ By considering the correlation of wavelet coefficients across scales, Portilla et al.  \cite{blsgsm} proposed to use Gaussian Scale Mixtures for image modeling and achieved promising denoising performance. It is widely accepted that natural image gradients exhibit heavy-tailed distributions \cite{weiss}, and the total variation (TV) based methods \cite{rudin1992nonlinear,osher2005iterative} actually assume Laplacian distributions of image gradients for denoising. The Fields of Experts (FoE) \cite{foe} proposed by Roth and Black models the filtering responses with Student's t-distribution to learn filters through Markov Random Field (MRF) \cite{Bishop}. Recently, Schmidt and Roth proposed the cascade of shrinkage fields (CSF) to perform denoising efficiently \cite{csf}.

Instead of modeling the image statistics in some transformed domain (e.g., gradient domain, wavelet domain or filtering response domain), another popular approach is to model the image priors on patches. One representative is the sparse representation based scheme which encodes an image patch as a linear combination of a few atoms selected from a dictionary \cite{olshausen1996emergence,olshausen1997sparse,ksvd}. The dictionary can be chosen from the off-the-shelf dictionaries (e.g., wavelets and curvelets), or it can be learned from natural image patches.\ The seminal work of K-SVD \cite{ksvdtsp,ksvd} has demonstrated promising denoising performance by dictionary learning, which has yet been extended and successfully used in various image processing and computer vision applications \cite{srcolor,srcvpr,lcksvd}. By viewing image patches as samples of a multivariate variable vector and considering that natural images are non-Gaussian, Zoran and Weiss \cite{epll,gmmnips} and Yu et al.  \cite{ple} used Gaussian Mixture Model (GMM) to model image patches, and achieved state-of-the-art denoising and image restoration results, respectively.


\section{Realistic Color Image Denoising}
\label{sec:review:feature}

During the last decade, a few methods have been proposed for real color image denoising.\ Among them, the CBM3D method \cite{cbm3d} is a representative one, which first transforms the RGB image into a luminance-chrominance space (e.g., YCbCr) and then applies the benchmark BM3D method \cite{bm3d} to each channel separately.\ The non-local similar patches are grouped by the luminance channel.\ In \cite{Liu2008}, Liu et al.\ proposed the ``Noise Level Function'' to estimate and remove the noise for each channel in natural images.\ However, processing each channel separately would often achieve inferior performance to processing the color channels jointly \cite{mairal2008sparse}.\ Therefore, the methods \cite{noiseclinic,ncwebsite,Zhu_2016_CVPR} perform real color image denoising by concatenating the patches of RGB channels into a long vector.\ However, the concatenation treats each channel equally and ignores the different noise statistics among these channels.\ The method in \cite{crosschannel2016} models the cross-channel noise in real noisy images as multivariate Gaussian and the noise is removed by the Bayesian non-local means filter \cite{kervrann2007bayesian}.\ The commercial software Neat Image \cite{neatimage} estimates the noise parameters from a flat region of the given noisy image and filters the noise accordingly.\ The methods in \cite{crosschannel2016,neatimage} ignore the non-local self-similarity of natural images \cite{bm3d,wnnm}. 