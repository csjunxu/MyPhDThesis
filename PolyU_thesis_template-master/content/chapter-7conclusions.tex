% !TEX root = ../thesis.tex
%
\chapter{Conclusions}
\label{sec:conclusions}


\section{Section 1}
\label{sec:conclusions:sec1}
In this thesis, I mainly studied the image denoising problem. 

In the first work, I proposed a patch group prior based image denoising (PGPD) method. The proposed PGPD method extended existing patch based learning scheme to patch group based learning scheme. This extension improve the patch based denoising method by considering the non-local self similarity (NSS) property of natural images. This NSS property can make the proposed PGPD method maintains image detail better than patch based methods such as EPLL \cite{epll}. Experiments demonstrate that the proposed PGPD method is both effective and efficient on denoising images corrupted by synthetic additive white Gaussian noise (AWGN).

In the second work, I proposed to extend the proposed patch group prior to realistic image denoising problem. I firstly employ the GMM model to learn the patch group prior and then employ the learned prior to guide the dictionary learning for the input noisy image. The learned dictionary can be more adaptive to the input image and hence can be used to remove the nosie in the input image.

In the third work, I proposed a multi-channel model for real color image denoising. The multi-channel scheme helps to consider the different noise statistics in different channels of the sRGB images. Since the weighted nuclear norm minimization (WNNM) model has already achieved state-of-the-art performance on the AWGN noise removal, I extended the WNNM model and the proposed Multi-channel WNNM (MCWNNM) model achieved state-of-the-art performance on both synthetic and real color image denoising problems.





\section{Future Work}
\label{sec:conclusions:future}

