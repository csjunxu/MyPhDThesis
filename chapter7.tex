% !TEX root = ../XJ_thesis.tex
%
\chapter{Conclusions and Future Work}
\label{sec:conclusions}


\section{Conclusions}
\label{sec:conclusions:sec1}
In this thesis, we proposed several image denoising methods for both synthetic additive white Gaussian noise (AWGN) and real-world noise. These methods can be categorized into external prior based, hybrid prior based, and internal prior based methods.

We firstly proposed an external prior based denoising method for synthetic AWGN. The proposed Patch Group Prior based image Denoising (PGPD) method extended existing patch based learning scheme \cite{epll} to patch group based learning scheme. This extension improve the patch based denoising method by considering the nonlocal self-similarity (NSS) prior of natural images. Extensive experiments demonstrated that the proposed PGPD not only achieves highly competitive PSNR results with state-of-the-art denoising methods, but also is highly efficient and preserves better the image edges and textures.

Then we proposed a hybrid prior based method for the real-world image denoising problem. We employed the Gaussian Mixture Model (GMM) to learn the NSS prior and utilized the learned NSS prior to guide the subspace clustering and dictionary learning of the input real-world noisy image. Extensive experiments on three real-world noisy image datasets, including a new dataset constructed by us by different types of cameras and camera settings, demonstrated that our proposed Guided method achieves much better performance than state-of-the-art image denoising methods in terms of both quantitative measure and perceptual quality.

To fully exploit the internal NSS prior, we proposed a multi-channel model for real color image denoising. The multi-channel scheme considers different noise statistics in different channels of the real-world color images. We introduced a weight matrix to the data term in the RGB channel concatenated weighted nuclear norm minimization (WNNM) model, and the resulting MC-WNNM model can process adaptively the different noise in RGB channels. Extensive experiments on synthetic and real-world noisy image datasets demonstrated that the proposed MC-WNNM method outperforms significantly the other competing denoising methods.

Besides, we proposed a novel trilateral weighted sparse coding (TWSC) scheme to exploit the noise properties across different channels and local patches.\ Specifically, we introduced two weight matrices into the data term of sparse coding model to adaptively process each patch in each channel, and a weight matrix to better model image priors.\ The proposed TWSC model was solved via the ADMM algorithm and the solution existence and convergence can be guaranteed.\ Experiments demonstrated the superior performance of TWSC to the state-of-the-art denoising methods, including those methods designed for realistic noise.

In the final work, we constructed a new real-world noisy image dataset. Specifically, we collected 40 images of diffrent scenes. The constrcuted dataset is called PolyU Realistic Image dataset, on which we evaluate the state-of-the-art image denoising methods and the proposed methods in this thesis. We evaluated these competing methods on two existing benchmark datasets and our proposed dataset. Extensive experiments demonstrated that the proposed TWSC, MCWNNM, and Guided methods achieve better denoising performance than the previous methods on PSNR, SSIM, and visual quality due to more consideration of the real-world noise statistics. Based on our analysis, the real-world image denoising problem is still a challenging task and need future research for new real-world image denoising methods.

\section{Future Work}
\label{sec:conclusions:future}

In the future work, we can expand the frontiers of the works in this thesis in the following directions:

\begin{itemize}
\item We need formulate better mathematical models for the real-world noise and study its statistical properties. This is still an open question which need more investigations.

\item We need more robust methods for image and noise prior learning. This will make our imaging systems more reliable to the corruption of the noise in real-world applications.

\item With the increase of real-world noisy images, it is useful to develop new discriminative learning denoising methods for real-world image denoising problem. According to whether the ``ground truth'' noise-free images can be provided or not, these discriminative methods can be divided into two categories: methods with paired samples and methods without paired samples. 

\item How to evaluate the visual quality of the denoised images is another problem which need more investigations.

\item It is essential to develop visual quality oriented image denoising methods. Besides, task driven (e.g., semantic segmentation) based image denoising methods are also very important for future imaging systems.

\end{itemize}

We will investigate these research directions in our future work.
