% !TEX root = ../XJ_thesis.tex
%
\chapter{Conclusions and Future Work}
\label{sec:conclusions}


\section{Conclusions}
\label{sec:conclusions:sec1}
In this thesis, we proposed several image denoising methods for both synthetic additive white Gaussian noise (AWGN) and realistic noise in real-world noisy images. These methods can be categorized into external prior based, hybrid prior based, and internal prior based methods.

We firstly proposed an external prior based denoising method for synthetic AWGN. The so-called Patch Group Prior based Denoising (PGPD) method extended existing patch based learning scheme \cite{epll} to patch group based learning scheme by modeling the nonlocal self-similarity (NSS) prior of natural images. Extensive experiments demonstrated that PGPD not only achieves highly competitive PSNR results with state-of-the-art denoising methods, but also is highly efficient and preserves better the image edges and textures.

Then we proposed a hybrid prior based method for the real-world image denoising problem. We employed the Gaussian Mixture Model (GMM) to learn the NSS prior and utilized the learned NSS prior to guide the subspace clustering and dictionary learning of the input real-world noisy image. Extensive experiments on three real-world noisy image datasets demonstrated that our proposed Guided method achieves much better performance than state-of-the-art image denoising methods in terms of both quantitative measurement and perceptual quality.

To fully exploit the internal NSS prior, we proposed a multi-channel model for real-world color image denoising. The multi-channel scheme considers different noise statistics in different channels of the real-world color images. We introduced a weight matrix to the data term in the RGB channel concatenated weighted nuclear norm minimization (WNNM) model, and the resulting MC-WNNM model can process adaptively the different noise in RGB channels. Extensive experiments on synthetic and real-world noisy image datasets demonstrated that the proposed MC-WNNM method outperforms significantly previous image denoising methods, including those methods designed for realistic noise.

We further proposed a novel trilateral weighted sparse coding (TWSC) scheme to exploit the noise properties across different channels and local patches.\ Specifically, we introduced two weight matrices into the data term of the traditional sparse coding model to adaptively process each patch in each channel, and a weight matrix to better model image priors.\ The proposed TWSC model was solved via the ADMM algorithm and the solution existence and convergence can be guaranteed.\ Experiments demonstrated the superior performance of TWSC to the state-of-the-art denoising methods, including those methods designed for realistic noise.

In the final work, we constructed a new real-world noisy image dataset. Specifically, we collected images of 40 different scenes. We evaluated the many denoising methods, including those proposed in this thesis, on two existing benchmark datasets and our proposed dataset. Extensive experiments demonstrated that the proposed TWSC, MCWNNM, and Guided methods achieve much better denoising performance than the previous methods on PSNR, SSIM, and visual quality. Based on our analysis, the real-world image denoising problem is still a challenging task and needs future research for new solutions.

\section{Future Work}
\label{sec:conclusions:future}

In the future work, we will expand the frontiers of the works in this thesis in the following directions:

\begin{itemize}
\item We need formulate better mathematical models for the real-world noise and study its statistical properties. This is still an open question which need more investigations.

\item We need more robust methods for the learning of image and noise priors. This will make the imaging systems, which are embedded with these methods, more reliable to the corruption of realistic noise in real-world applications.

\item According to whether the ``ground truth'' noise-free images is available or not, the potential discriminative methods can be divided into two categories: methods with paired noisy and ``ground truth'' images and methods without paired noisy and ``ground truth'' images. With the increase of real-world noisy images, it is useful to develop new discriminative learning denoising methods for the real-world image denoising problem. 

\item How to evaluate the visual quality of the denoised images in real-world image denoising problem is another question which needs more investigations.

\item It is essential to develop visual quality oriented (real-world) image denoising methods. Besides, task driven (e.g., semantic segmentation) based image denoising methods are also very important for future imaging systems.

\end{itemize}

We will investigate these research directions in our future work.
