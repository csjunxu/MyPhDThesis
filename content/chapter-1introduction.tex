% !TEX TS-program = pdflatex
% !TEX root = ../thesis.tex
%
\chapter{Introduction}
\label{sec:intro}

\cleanchapterquote{"Mens cujusque is est Quisque" – "Mind Makes the Man"}{Samuel Pepys}{}


Nowadays, cameras are becoming more and more widely used in many aspects of human lifes such as taking pictures, medical analysis, security monitoring and control, etc. The camera imaging pipelines are of particular importance since it is the key step of transforming the real scenes into the pictures or videos. However, during the imaging process, the noise is unavoidable to be generated due to many reasons.

\section{The Camera Imaging Pipeline}
\label{sec:intro:general}

The cameras capture the images and store as raw image formats. During the camera imaging pipeline, the photons are transformed into electronics by the photodiode in the camera sensor. The original sensor arrat (also called color filter array, or CFA) contains red, green, and blue channels, and these incomplete channels are transformed into the final RGB files via the raw converter. The camera imaging pipeline includes multiple stages such as reading raw image, black light subtraction, lens correction, demosaicing, noise reduction, white balancing, gamma curve, final color space conversion, etc \cite{browneccv2016}. Basically, a camera imaging pipeline includes demosaicing, white balancing and color space transform, gamut mapping, tone mapping, and JPEG compression \cite{crosschannel}. However, different cameras have varying structures and camera parameters, and hence resulting different imaging effects. Recently, there also exists learning based imaging pipelines which directly learn the  natural image priors from the RGB and raw images pairs.


\section{The Image Noise}
\label{sec:intro:current}

During the imaging pipeline, the noise will be generated. The key reason of noise generation is unstable measurement from the discrete nature of light and the thermal agitation. 

The major sources of noise generated during the imaging pipeline are the random noise , the spatial non-uniformity, and quantization noise. The random noise includes photon shot noise, dark current, and readout noise. The spatial non-uniformity noise includes the fixed pattern noise (PRNU, DCNU), CCD/CMOS specific noise. 


A simplified model including various noise sources (for each pixel):
\begin{equation}
\bm{P} = f((g_{cv}(\bm{C}+\bm{D})+\bm{N}_{reset})g_{out}+\bm{N}_{out})+\bm{Q}.
\end{equation}
Now the above equation is explained in details. 
$\bm{P}$ is the raw pixel value, 
$\bm{C}$ is the number of absorbed electrons (charges) transformed from the photons via the photon-diodes in the camera sensor,
$\bm{D}$ is the number of absorbed electrons generated by dark current,
$g_{cv}$ is the equivalent capacitance (EC) of the photo-diode,
$\bm{N}_{reset}$ is the thermal noise generated by the readout circuitry (or reset noise related to reset voltage),
$g_{out}$ is the gain factor during voltage to pixel value conversion (readout),
$\bm{N}_{out}$ is the readout noise,
$f$ is the camera response function, usually a linear function before attaining a saturation threshold,
$\bm{Q}$ is the quantization error happened during rounding to interger values. The quantization noise is normally negligible compared to the readout noise.

\section{The Proposed Methods}
\label{sec:intro:new}




\section{Thesis Structure}
\label{sec:intro:structure}


\textbf{Chapter \ref{sec:2review}} \\[0.2em]





\textbf{Chapter \ref{sec:3external}} \\[0.2em]




\textbf{Chapter \ref{sec:feature}} \\[0.2em]





\textbf{Chapter \ref{sec:real}} \\[0.2em]





\textbf{Chapter \ref{sec:conclusions}} \\[0.2em]





