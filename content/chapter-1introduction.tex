% !TEX TS-program = pdflatex
% !TEX root = ../thesis.tex
%
\chapter{Introduction}
\label{sec:intro}

\cleanchapterquote{"Mens cujusque is est Quisque" – "Mind Makes the Man"}{Samuel Pepys}{}


Nowadays, cameras are becoming more and more widely used in many aspects of human lifes such as taking pictures, medical analysis, security monitoring and control, etc. The camera imaging pipelines are of particular importance since it is the key step of transforming the real scenes into the pictures or videos. However, during the imaging process, the noise is unavoidable to be generated due to many reasons.

\section{The Camera Imaging Pipeline}
\label{sec:intro:general}

The camerascapture the images and store as raw image formats. During the camera imaging pipeline, the photons are transformed into electronics by the photodiode in the camera sensor. The original sensor arrat (also called color filter array, or CFA) contains red, green, and blue channels, and these incomplete channels are transformed into the final RGB files via the raw converter. The camera imaging pipeline includes multiple stages such as reading raw image, black light subtraction, lens correction, demosaicing, noise reduction, white balancing, gamma curve, final color space conversion, etc.


\section{The Image Noise}
\label{sec:intro:current}



\section{The Proposed Methods}
\label{sec:intro:new}




\section{Thesis Structure}
\label{sec:intro:structure}


\textbf{Chapter \ref{sec:2review}} \\[0.2em]





\textbf{Chapter \ref{sec:3external}} \\[0.2em]




\textbf{Chapter \ref{sec:feature}} \\[0.2em]





\textbf{Chapter \ref{sec:real}} \\[0.2em]





\textbf{Chapter \ref{sec:conclusions}} \\[0.2em]





