% !TEX TS-program = pdflatex
% !TEX root = ../thesis.tex
%

\chapter{Introduction}
\label{sec:intro}

%\cleanchapterquote{"Mens cujusque is est Quisque" – "Mind Makes the Man"}{Samuel Pepys}{}


Nowadays, CCD or CMOS cameras are becoming more and more important in many aspects of human life such as photography, artificial intelligence, and security system. For each camera product, the camera imaging pipeline in the camera is of particular importance since it is the core part to transform the photons reflected by the real scene being captured in the camera sensor into the pixel values of an image, which can be displayed on a screen. During the camera imaging process, the noise is unavoidablely generated due to many reasons. Two major reasons of noise generation are the discrete nature of light and the thermal agitation, which can cause the photon shot noise and the dark-current noise, respectively. Image denoising is the problem of recovering the laten clean image from the captured noise version. 

\textbf{Chapter abstract} This chapter will describe the image noise and its acquirezation equation, the image denoising problem, the reasons why image denoising is so important. I will also discuss the objective measures to evaluate the image denoising performance. Finally, I will summarize the contribution made by his thesis.
%\section{The Camera Imaging Pipeline}
%\label{sec:intro:general}

%The cameras capture the images and store as raw image formats. During the camera imaging pipeline, the photons are transformed into electronics by the photodiode in the camera sensor. The original sensor arrat (also called color filter array, or CFA) contains red, green, and blue channels, and these incomplete channels are transformed into the final RGB files via the raw converter. The camera imaging pipeline includes multiple stages such as reading raw image, black light subtraction, lens correction, demosaicing, noise reduction, white balancing, gamma curve, final color space conversion, etc \cite{browneccv2016}. Basically, a camera imaging pipeline includes demosaicing, white balancing and color space transform, gamut mapping, tone mapping, and JPEG compression \cite{crosschannel}. However, different cameras have varying structures and camera parameters, and hence resulting different imaging effects. Recently, there also exists learning based imaging pipelines which directly learn the  natural image priors from the RGB and raw images pairs.


\section{The Image Noise}
\label{sec:intro:current}

The realistic noise are very complex in real photography images captured by the camera sensors. One major reason is that the noise is unable to be explicitly modeled by some definitive probabilistic distributions. Often the sRGB images we look at the screen have been processed via the camera imaging pipeline, during which the noise will be much more complex than its initial status. 

The noise are generated mainly due to the discrete nature of light and the thermal agitation, which will cause unstable measurement in camera sensors. The major types of noise generated during the imaging pipeline are the random noise, the spatial non-uniformity noise, and quantization noise. The random noise includes photon shot noise, dark current, and readout noise. The spatial non-uniformity noise includes the fixed pattern noise (PRNU, DCNU), CCD/CMOS specific noise. 

To better describe the noise quantitatively, we provide a simplified signal acquisition model \cite{} including various noise sources (for each pixel) as follows:
\begin{equation}
\label{e11}
\bm{P} = f((g_{cv}(\bm{C}+\bm{D})+\bm{N}_{reset})g_{out}+\bm{N}_{out})+\bm{Q}.
\end{equation}
The above equation is explained in details: 
\begin{itemize}
\item $\bm{P}$ is the raw pixel value;
\item $f$ is the camera response function, usually a linear function before attaining a saturation threshold;
\item $\bm{C}$ is the number of absorbed electrons (charges) transformed from the photons via the photon-diodes in the camera sensor, which can be modeled by a Poisson distribution;
\item $\bm{D}$ is the number of absorbed electrons generated by dark current by thermal generation, which is often modeled by a Poisson distribution;
\item $\bm{N}_{reset}$ is the thermal noise generated by the readout circuitry (or reset noise related to reset voltage), which can be well modeled by a Gaussian disribution;
\item $\bm{N}_{out}$ is the readout noise, which is also modeled by a Gaussian distribution;
\item $\bm{Q}$ is the quantization error happened during rounding to interger values, usually uniformly distributed and normally negligible compared to the readout noise;
\item $g_{cv}$ is the equivalent capacitance (EC) of the photo-diode and the gain factor during charge to voltage conversion;
\item $g_{out}$ is the gain factor during voltage to pixel value conversion (readout).
\end{itemize}
After some merging and simplifying, the acquisition model of the signal can be formulated as follows:
\begin{equation}
\label{e12}
\begin{split}
\textbf{P} 
&=f((g_{cv}(\textbf{C}+\textbf{D})+\textbf{N}_{reset})g_{out}+\textbf{N}_{out})+\textbf{Q},
\\
&=f(g_{cv}g_{out}(\textbf{C}+\textbf{D})+g_{out}\textbf{N}_{reset}+\textbf{N}_{out})+\textbf{Q},
\\
&=f(g\lambda+N_{R})+\textbf{Q},
\end{split}
\end{equation}
where $g = g_{cv}g_{out}$ is the overall camera gain factor, $\lambda=\textbf{C}+\textbf{D}$ is \# of electrons in pixel capacitor and also Poisson distributed, since ``Sum of Independent Poisson Random Variables is also Poisson'', and $N_{R}=g_{out}\textbf{N}_{reset}+\textbf{N}_{out}$ is the overall readout noise and also Gaussian distributed, since ``Sum of Independent Gaussian Random Variables is also Gaussian''. In summary, the overall noise before the camera imaging pipeline can be modeled as a mixed Poisson and Gaussian distribution \cite{}.


Though can be approximated as mixed Gaussian and Poisson distribution, the realistic noise will become more complex after the camera imaging pipeline, which has been analyzed in \cite{crosschannel}. Hence, the real-world noise is much more complex than the traditional additive white Gaussian noise, and should be paid more attention.

In order to obtain good images while still maintaining the structures and details of the captured scene, image denoising is a good choice. In this thesis, I will present my work on image denoising, including synthetic noisy images and real noisy images. Note that the realistic noise in real color noisy images are much more complex than the usually synthetic additive white Gaussian noise (AWGN). This is because that during the camera imaging pipeline, the noise generated will be modified to be more complex than the noise in the raw sensor data. Hence, in this thesis, we pay more attention to the realistic noise generated during the camera imaging pipeline. 

In image denoising community, the most commonly studied noise is the additive white Gaussian noise, which is used to model the independent noise in the raw images. The AWGN noise is described as a Gaussian distribution $\mathcal{N}(0,\sigma^{2})$, which means that the noise is Gaussian distributed with $0$ mean and $\sigma$ standard deviation. Most of methods are focus on this type of noise since it is a good testing bed for many other image restoration problems such as super-resolution, deblurring, inpainting, etc.

\section{Problem Formulation: Image Denoising}

\section{The Proposed Methods}
\label{sec:intro:new}

This thesis is made up of several work I have done during my study, in which I focus on designing new and better image denoising algorithms. 

Firstly, we propose a method for denoising synthetic AWGN noise, from which we can study the performance of the nonlocal self-similarity priors of natural images. In fact, we propose to learn the external NSS priors and apply the learned model on denoising AWGN noise. The proposed method achieves state-of-the-art performance on AWGN denoising on both effectiveness and efficiency. 

Basing on the success on the synthetic noise removal, I propose to exploit the power of the NSS priors in natural images to deal with the complex realistic noise in real-world noisy images. Specifically, we propose three methods exploiting the NSS priors of natural images for real noisy image denoising, which can be introduced as follows.

In the first method, I propose to learn the NSS prior from the external natural images, and then apply the learned external prior to guide the learning of the internal NSS prior of the input real noisy image. The experiments on two commonly used datasets and a new one we constructed to implement the shortage of existing datasets, demonstrate that the proposed method can achieve better performance than existing color image denoising methods such as CBM3D \cite{cbm3d}, the state-of-the-art Gaussian noise removal methods \cite{bm3d,mlp,csr}, and the real noisy image denoising methdos \cite{} including a commercial software Neat Image \cite{neatimage}, which is embeded into the famous PhotoShop CS for image processing tasks.


In the second method, I propose to employ the low rank model describe fully the the internal NSS prior, basing on the observed fact that the similar image patches can be contanated as a matrix of low rank. Different from the previous work, I extend the WNNM model and apply it to multi-channel version to make it feasible for color image denoising. 

In the third method,  is to use the sparse coding based method with additional weighting scheme to regard the local noise in real noisy images as a Gaussian and the prior is used to deal with the real noisy image.

Finally, to make my thesis more comprehensive, I construct a large benchmark of real noisy images captured by different types of famous commercial cameras, on which I also evaluate the image denoising methods mentioned above and the proposed methods in this thesis. 




The structure of this thesis is organized as follows: in the chapter 2, we review the literatures in the image denoising area; in the chapter 3, we introduce the fully external method; in the chapter 4, we introduce the external prior guided internal method; in the chapter 5, we introduce the internal method based on low ran model; in the chapter 6, we introduce the internal method based on sparse coding model; in the chapter 7, we introduce the real noisy image dataset we construct, and finaly evaluate the proposed methods with the compared competing methods, both for synthetic AWGN or Poisson noise and real noise, including the commercial software designed especially for real noise. 


\section{Thesis Structure}
\label{sec:intro:structure}


\textbf{Chapter \ref{sec:review}: Literature Review} \\[0.2em]

In this chapter, we review the related work and give a detailed introduction of the literature. We will first review the most representative work on additive white Gaussian noise removal. I review the detailed work on camera imaging pipeline and realistic noise generated in the camera sensors. I will also review the work on real noisy image denoising.



\textbf{Chapter \ref{sec:external}: External Nonlocal Self-Similarity Prior Learning for Synthetic Gaussian Noise Removal} \\[0.2em]

In this chapter, I will introduce our work on external nonlocal self-similarity (NSS) prior learning for synthetic Gaussian noise removal. As far as we know, this work is the first to learn the NSS priors of natural clean images, while previous work only utilize the NSS priors of input noisy image for online denoising. The advantages of this offline learning is that it can preserve the details of natural images while being much faster then most online denoising methods.


\textbf{Chapter \ref{sec:guided}: External Prior Guided Internal Prior Learning for Real Noisy Image Denoising} \\[0.2em]

In this chapter, I will introduce our work on external prior guided internal prior learning method for real noisy image denoising. This work can maintain the advantages of both sides: from the external perspective, the method can preserve the structures of natural images better than the internal methods, while from the perspective of internal method, the proposed method can recover the details of the input noisy image better than the external methods.



\textbf{Chapter \ref{sec:internallr}: Multi-channel Weighted Nuclear Norm Minimization for Real Color Image Denoising} \\[0.2em]

In this chapter, we introduce a multi-channel weighted nuclear norm minimization (MC-WNNM) method. This method regards different channels in RGB images differently to adaptively process the real color noisy images. Besides, this work also propose a new strategy for color image denoising. Experiments demonstrate that the proposed method can achieve better performance on real color image denoising than existing state-of-the-art methods, including some commercial software.



\textbf{Chapter \ref{sec:internalsc}: A Triple Weighted Sparse Coding Scheme for Realistic Noisy Image Denoising} \\[0.2em]

In this chapter, I introduce a novel sparse coding based method for real color image denoising. In this method, I regard the noise in each of the local region in the real noisy image as a Gaussian, and propose a triplely weighted scheme to deal with the complex realistic noise in real color noisy images. Experiments show that the proposed method performs better and faster than the nuclear norm based method mentioned in previous chapter.


\textbf{Chapter \ref{sec:dataset}: A Benchmark on Real Color Noisy Image, with Comprehensive Evaluation of State-of-the-art} \\[0.2em]

To fully boost the research of real color noisy image denoising, we construct a large benchmark on real color noisy images. This dataset is collected from several representative cameras with comprehensive settings on contents, lighting, ISO, shutter, and aperture, etc. Based on this newly established dataset, we fully evaluated existing denoising methods, including the methods designed for synthetic Gaussian noise and the methods designed especially for real color noise. We believe that this new dataset will largely boost the research of the image denoising especially the realistic image denoising problems.
















