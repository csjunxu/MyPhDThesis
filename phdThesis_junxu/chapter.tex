\chapter{Image Segmentation by Iterated Region Merging with Localized Graph Cuts} \label{chap:IGC}
\section{Introduction} \label{sec:IGC:Introduction}
In this chapter we consider the most common type of interactive segmentation: segmenting the object of interest from its background. In the past few years, various approaches to interactive segmentation have been proposed. For example, livewire \cite{Falaco98} allows the user to interactively select certain pixels where the segmentation boundary should pass. However, high complexities of object shapes (e.g. intricate shapes with lots of protrusions and indentations) might lead to many interactions for an acceptable segmentation. And images with the large size will require more computational time. To obtain real time response to the user's actions, independent of the image size, Falc{\~{a}}o \cite{Falco} proposed a modified livewire method, which exploits three properties of Dijkstra's algorithm to compute minimum-cost paths in sub-linear time. Active contour, or snake \cite{Kass88}, is defined as an energy-minimizing spline. After initializing the contour close to the original object boundary, the contour will fit the actual object boundary iteratively. Level sets based segmentation method \cite{Osher88} uses implicit active contour models, in which the numerical computation involving curves and surface is performed without having to parameterize the objects.

Another preferable interactive segmentation method based on combinatorial optimization is graph cuts \cite{Yuri01,Yuri06}. It addresses segmentation in a global optimization framework and guarantees a globally optimal solution to a wide class of energy functions. In addition, the user interface of graph cuts is convenient - seeds can be loosely positioned inside the object and background regions, which is easier compared to placing seeds exactly on the boundary, like in livewire \cite{Falaco98}. Because graph cuts can involve a wide range of visual cues, a number of recent literature further extended the original work of Boykov and Jolly \cite{Yuri01} and developed the use of regional cues
\cite{grabcut,stereo}, geometric cues \cite{Yuri_geodesics,Yuri_what}, shape cues \cite{shape2,shape3}, stereo cues \cite{stereo}, or even topology priors \cite{topology} as global constraints in the graph cuts framework. When foreground and background color distributions are not well separated, the traditional graph cuts \cite{Yuri01} can not achieve satisfying segmentation. Some advanced versions of graph cuts are developed \cite{grabcut,lazy,paint,Peng}, which are more robust and substantially simplify the user interaction. In \cite{lazy}, the user interaction can be applied on both coarse and fine scales. This work inherits the advantages in region and boundary based methods for image segmentation. The work proposed in \cite{paint} makes a progressive local selection on the object of interest. Instant visual feedback is provided to the user for a quick and effective image editing.

In the classical graph-based framework, most of segmentation methods consider pixels or groups of pixels as the nodes in a graph. The edge weight estimation usually takes into account image attributes, for example color, gradient and texture. An efficient edge weight assignment method was proposed by Miranda et al. \cite{R1}, where the object information obtained from user interaction as well as the
image attributes are both used for estimating edge weights. Separating from the image segmentation process, it can act as a basic step for high accuracy image segmentation. Some other works studied graph structures for designing image processing operators. Image foresting transform (IFT) \cite{R3, R4}, for example, defines a minimum-cost path forest in a graph, and provides a mathematically sound framework for many image processing operations. Based on similar graphs, a theoretical analysis between optimum-path forests and minimum cut was given in \cite{R5}. Under some conditions, the two algorithms were proven to produce the same result.

In our preliminary work \cite{Peng}, we explore the graph cuts algorithm by extending it to a region merging scheme. Starting from seed regions given by the user, graph cuts is conducted on a
propagated sub-graph where the regions are regarded as the nodes of the graph. An iterated conditional mode (ICM) is studied and the maximum a posterior (MAP) estimation is obtained by virtual of graph
cuts on each growing sub-graph. The segmentation process is stopped when all the regions are labeled. In \cite{Peng}, the initial segmentation is obtained by Mean-shift algorithm, which is a
sophisticated segmentation technique.  While in this chapter, the initial segmentation is obtained by the simple watershed algorithm \cite{watershed}. In each iteration, a semi-supervised algorithm is
applied to learn a classifier. Consequently, the most confident labels will contribute for new seed regions in the next iteration.

The proposed method is a novel extension of the standard graph cuts
algorithm. Rather than segmenting the entire image all at once, the
segmentation is performed incrementally. It has many advantages to
do this. First of all, using sub-graph significantly reduces the
complexity of background content in the image. The many unlabeled
background regions in the image may have unpredictable negative
effect on graph cuts optimization. This is why the global optimum
obtained by graph cuts often does not lead to the most desirable
result. However, by using a sub-graph and blocking those unknown
regions far from the labeled regions, the background interference
can be much reduced, and hence better results can be obtained under
the same amount of user interaction. Second, the algorithm is run on
the sub-graph that comprises object/background regions and the
surrounding un-segmented regions, thus the computational cost is
significantly less than running graph cuts on the whole graph which
is based on image pixels. Third, as a graph cuts based region
merging algorithm, our method obtains the optimal segmentation on
each sub-graph. In interactive image segmentation, user input
information helps to enhance the discontinuities between object and
background by constructing color data models \cite{grabcut}, which
represent object and background respectively. Some simple methods
such as color histograms can be used to calculate these models. In
this work, the construction of the object and the background color
models are obtained from the most confident labels by a learned
classifier. This scheme automatically collects more reliable
information for the next round of segmentation.

Although the user input is helpful in steering the segmentation
process to reduce the ambiguities, too much interaction will lead to
a tedious and time-consuming work. If the object is in a complex
environment from which the background can not be trivially
subtracted, a significant amount of interaction may be required.
Moreover, the complex content of an image also makes it hard to
provide user guidance for accurate segmentation while keeping the
interaction as less as possible. Therefore, some algorithms allow
further user edit based on the previous segmentation results
\cite{grabcut,lazy,paint,Video} until the desired result is
achieved. In comparison to the traditional graph cuts algorithm, the
proposed method is able to reduce the amount of user interaction
needed for a desirable segmentation result, or that given a fixed
amount of user interaction it increases the quality of the final
segmentation result. Experiments show that with poor initialization
(i.e. user inputs), the segmentation results of standard graph cuts
algorithm might be far from what we expect, while the proposed
method can still offer good results. In addition, much better
segmentation results can be achieved by the proposed method for
images with complex background.

\section{Image Segmentation by Graph Cuts}\label{sec:2}
Image segmentation can be naturally taken as a labeling problem. Given a set of labels $L$ and a set of sites $S$ (e.g, image pixels or regions),our goal is to assign each of the sites $p\in S$ a label $f_p\in L$. The graph cuts framework proposed by Boykov and Jolly~\cite{Yuri01} addresses the segmentation on binary images, which solves a labeling problem with two labels. The label set is $L=\{0,1\}$, where 0 corresponds to the background and 1 corresponds to the object. Therefore, labeling is a mapping from $S$ to $L$ and the label assignments to all pixels are denoted by $f=\{f_p|f_p\in L\}$. An energy function in a ``Gibbs'' form is formulated as:
 \begin{equation}
      E(f) = E_{data}(f)+\lambda E_{smooth}(f) \label{IGC:equ:energy1}
    \end{equation}
The data term $E_{data}$ consists of constraints from the observed
data and measures how sites like the labels that $f$ assigns to
them. It is usually defined to be:
   \begin{equation}
     E_{data}(f) = \sum_{p\in S}D_p(f_p)
   \end{equation}
   where $D_p$ measures how well label $f_p$ fits site $p$. For
    example, we can use intensities of marked sites (seeds) to learn the
    histograms for the object and the background intensity distributions
    $Pr(I|"obj")$ and $Pr(I|"bkg")$. Then $D_p$ can be expressed as
    follows:
    \begin{equation}
    D_{p}("obj")=-ln Pr(I_p|"obj")
    \end{equation}
    and
    \begin{equation}
    D_{p}("bkg")=-ln Pr(I_p|"bkg")
    \end{equation}
    $D_p$ is the penalty of assigning the label $f_p$ to site $p\in
    S$. The negative log-likelihoods should be small if $p$ likes $f_p$ and vice versa.
    $E_{smooth}$ is called the smoothness term and measures the
    extent to which $f$ is not piecewise smooth. The typical form of
    $E_{smooth}$ is:
    \begin{equation}
    E_{smooth}=\sum_{\{p,q\}\in \mathcal{N}}V_{pq}(f_p,f_q)
    \end{equation}
    where $\mathcal{N}$ is a neighborhood system,
    such as a 4-connected neighborhood system or an 8-connected neighborhood system. The
    smoothness term typically used for image segmentation is the Potts Model \cite{Yuri98},
    which is
    \begin{equation}
    V_{pq}(f_p,f_q)=\omega_{pq} \times T(f_p \neq f_q)\label{IGC:equ:smoothness}
    \end{equation}
    where:
    \[
        T(f_p \neq f_q) = \left\{
        \begin{array}{ll}
        1 &\mbox{if $f_p \neq f_q$}\\
        0 &\mbox{otherwise}
        \end{array}
        \right.
    \]

    The model (\ref{IGC:equ:smoothness}) is a piecewise constant model because it encourages labelings consisting of several regions where sites in the same region have the same labels. In image segmentation, we want the boundary to lie on the intensity edges in the image. A typical choice for $\omega_{p,q}$ is as follows:
    \begin{equation}
    \omega_{pq}=e^{-\frac{|I_p-I_q|^2}{2\delta^2}}\cdot
    \frac{1}{dist(p,q)}
    \end{equation}
     For gray images, $I_p$ and $I_q$ are the intensities of site $p$ and $q$. For color images, they are replaced by the notations of  $\vec{I_p}$ and $\vec{I_q}$, which can be the LAB color vectors of sites $p$ and $q$.  $dist(p,q)$ is the distance between sites $p$ and $q$. Parameter $\delta$ is related to the level of variation between neighboring sites within the same object. The parameter $\lambda$ is used to control the relative importance of the data term versus the smoothness term. If $\lambda$ is very small, only the data term matters. In this case, the label of each site is independent from the other sites. If $\lambda$ is very large, all the sites will have the same label. Minimization of the energy function can be done using the min-cut/max-flow
    algorithm as described in~\cite{Yuri01}.

    Now we need to construct a graph corresponding to the energy  function (\ref{IGC:equ:energy1}). There are two additional nodes: the source terminal $s$ and the sink terminal $t$, representing the object and the background respectively. Each node in the graph is connected to $s$ and $t$ by two $t$-links. And each pair of neighboring nodes is connected by an $n$-link. The weights of $t$-links for seed pixels can be seen as hard constraint imposed on the segmentation. In initialization, the user will mark some pixels as the object or the background so that these pixels will keep their initial labels in the final result. If pixel $p$ is marked as an object label, the edge between $p$ and $s$ should be set to infinity and the edge between $p$ and $t$ should be set to zero.  $N$-links correspond to the penalty for discontinuity between the two neighboring pixels. They are derived from the smoothness term $E_{smooth}$ in energy function (\ref{IGC:equ:energy1}). And the weight of a $t$-link corresponds to a penalty for assigning the label to the pixel. It will be derived from the data term $E_{data}$ in the energy function (\ref{IGC:equ:energy1}).

\section{Iterated Region Merging with Localized Graph Cuts}\label{sec:3}
\subsection{Initial Segmentation by Modified Watershed Algorithm} \label{sec:3:algo}
    \begin{figure}[htp]
    \centering
    {\subfigure[]
    {\label{fig:1:seeds}\includegraphics[width=2in]{images/IGC/fig1_initial.pdf}}
    \subfigure[]
    {\label{fig:1:standard}\includegraphics[width=2in]{images/IGC/fig1_Standard.pdf}}}
    \caption{(a) Original image with user input seeds. The background seeds
    are in green, and object seeds are in red. (b) The segmentation results by standard graph
    cuts.  }
    \label{fig:standard}
    \end{figure}

    In the original graph cuts algorithm \cite{Yuri01}, the segmentation is directly performed on the image pixels. There are two problems for such a processing. First, each pixel will be a node in the graph so that the computational cost will be high; second, the segmentation result may not be smooth, especially along the edges. Fig.\ref{fig:standard} shows an example of the graph cuts segmentation result. It can be seen that although there should be clear boundary between the object
    and background, the graph cuts fails to give a smooth segmentation map by labeling some object pixels as background, or vice versa. Actually, in the early work of Wu and Leahy~\cite{Wu_minicut}, it was noticed that the minimum cut criteria favored cutting small sets of isolated nodes in the graph.

    To alleviate this problem, Veksler~\cite{shape3} included a shape constraint in the graph cuts
    energy function, which encourages a long object boundary. Some other segmentation criterions were also proposed to solve this problem, such as normalized cuts~\cite{Shi} and ratio cut~\cite{RatioCut}. In this chapter, we adopted a relatively simple but effective strategy to solve this problem by introducing some low level image processing techniques to graph cuts. In~\cite{lazy}, Li et al. used watershed~\cite{watershed} for initial segmentation to speed up the graph cuts optimization process in video segmentation. With such initialization, the image can be partitioned into many small homogenous regions, and then each region, instead of each pixel, is taken as a node in the graph. In this way the computational cost can be reduced significantly, while the object boundary can be better preserved. The watershed technique is also used with some modification.
    Watershed algorithm produces coherent over-segmented regions which preserve most structures
     of the interest object. However, the standard watershed algorithm is very sensitive to noise and thus leads to severe over-segmentation (see Fig.~\ref{fig:2:water1}). There are some
     edge-preserving smoothing techniques, such as median filtering, can help to reduce noise and trivial structures. Therefore, to reduce over-segmentation, we apply median filtering on the gradient image before conducting the watershed algorithm. Fig.~\ref{IGC:fig:watershed} shows an example. Fig.~\ref{fig:2:grad1} is the gradient image of the original image in  Fig.~\ref{fig:1:seeds} and Fig.~\ref{fig:2:water1} is the watershed segmentation of it. Clearly, there is a severe over-segmentation in Fig.~\ref{fig:2:water1}. Such small regions are not reliable for calculating the region statistics and they will also increase the computational cost in our region merging algorithm Fig.~\ref{fig:2:grad2} is the median filtering output of the gradient image in Fig.~\ref{fig:2:grad1}, and Fig.~\ref{fig:2:water2} is the watershed segmentation result on it. We see that the over-segmentation is significantly reduced, while the contour of the object is well preserved. Note that we can use more sophisticated initial segmentation techniques in the proposed method. To weaken the importance of initial segmentation, the watershed algorithm is adopted for its simplicity.

    \begin{figure}[htp]
    \centering
    \subfigure[]
    {\label{fig:2:grad1}\includegraphics[width=1.9in]{images/IGC/fig2_grad1.pdf}}
    {\subfigure[]
    {\label{fig:2:water1}\includegraphics[width=1.9in]{images/IGC/fig2_w1.pdf}}
    \subfigure[]
    {\label{fig:2:grad2}\includegraphics[width=1.9in]{images/IGC/fig2_grad2.pdf}}
    \subfigure[]
    {\label{fig:2:water2}\includegraphics[width=1.9in]{images/IGC/fig2_w2.pdf}}}
    \caption{Initial segmentation using modified watershed algorithm. (a) is the
    gradient image of Fig. \ref{fig:1:seeds}; (c) is the median filtering result of (a); (b) and
    (d) are the watershed segmentation results of (a) and (c) respectively. We see
    that the over-segmentation is significantly reduced in (d) compared with (b).}
    \label{IGC:fig:watershed}
    \end{figure}

\subsection{Iterated Conditional Mode}

Although graph cuts technique provides an optimal solution to the energy function~(\ref{IGC:equ:energy1}) for image segmentation, the complex content of an image makes it hard to precisely segment the whole image all at once. In the proposed region merging based segmentation algorithm, the one-shot minimum cut estimation algorithm is replaced by a novel iterative procedure, in which the object/background distributions are updated according to the previous segmentation results and new nodes are added until the whole image is segmented. This problem is studied in a way like the iterated conditional mode (ICM) proposed by Besag~\cite{ICM}, where the local conditional probabilities is maximized sequentially.

In computer vision, an image can be represented by a graph $G =<V,E>$, where $V$ is a set of nodes corresponding to image elements (e.g. pixels, regions), and $E$ is a set of edges connecting the pairs of nodes. We say two nodes are incident with an edge and that these nodes are adjacent or neighbors of each other. Edge weights of the graph are computed as the dissimilarity between the connected nodes (e.g. the distance of region histograms). A sub-graph $G'=<V',E'>$ can be defined such that $V'\subseteq V$ and $E'\subseteq E$, where $E'$ contains only edges built from the nodes of $V'$. In this chapter, we consider image regions as the graph nodes, and the neighborhood of a node in $V'$ corresponds to its adjacent regions in the image. Inspired by ICM, we consider the graph-cuts algorithm in a ``divide and conquer'' style: finding the minima on the sub-graph and extending the sub-graph successively until reach the whole graph. The proposed method works iteratively, in place of the previous one-shot graph cuts algorithm~\cite{Yuri01}.

Given the observed data $d_p$ of site $p$, the label $f_p$  of site $p$ and the set of labels $f_{S-\{p\}}$ which is at the site in $S$-$\{p\}$, where $f_p\in L$ and $S$-$\{p\}$ is the set difference. We sequentially assign each $f_i$ by maximizing conditional probability         $P(f_p|d_p,f_{S-\{p\}})$ under the MAP-MRF framework. There are two assumptions in calculating $P(f_p|d_p,f_{S-\{p\}})$. First, the observed data $d_1,\dots,d_m$ are conditionally independent given $f$ and that each $d_p$ depends only on $f_p$. Second, $f$ depends on labels in the local neighborhood, which is Markovianity, i.e. $P(f_p|d_p,f_{S-\{p\}})=P(f_p|f_{N_p})$, where $N_p$ is a neighborhood system of site $p$. Markovianity depicts the local characteristics
        of labeling. With the two assumptions we have:
            \begin{equation}
            P(f_p|d_p,f_{S-\{p\}})=\frac{P(d_p|f_p)\cdot P(f_p|f_{N_p})}{P(d)}
            \end{equation}
where $P(d)$ is a normalizing constant when $d$ is given. There is:
            \begin{equation}
            P(f_p|d_p,f_{S-\{p\}})\propto P(d_p|f_p)\cdot P(f_p|f_{N_p})
            \end{equation}
where $\propto$ denotes the relation of direct proportion. The posterior probability satisfies:
            \begin{equation}
            P(f_p|d_p,f_{S-\{p\}})\propto e^{-U(f_p|d_p,f_{N_p})}
            \end{equation}
where $U(f_p|d_p,f_{N_p})$ is the posterior energy and satisfies:
            \begin{eqnarray}
            U(f_p|d_p,f_{N_p})
                = U(d_p|f_p)+U(f_p|f_{N_p})\nonumber\\
                = U(d_p|f_p)+ \sum_{p'\in N_p}U(f_p|f_{p'})
            \end{eqnarray}
 $U(d_p|f_p)$ is the data term corresponding to function (\ref{IGC:equ:energy1}), and $\sum_{p'\in N_p}U(f_p|f_{p'})$ is the smoothness term which relates to the number of neighboring sites whose labels $f_{p'}$ differ from $f_p$. The MAP estimate is equivalently found by minimizing the posterior energy:
 \begin{equation}
  f^{k+1}=\arg\min_{f} U(f|d,f_{N}^{k})
 \end{equation}
where $f_N^k$ is the optimal labeling of graph nodes obtained in previous $k$ iterations. The labeling result in each iteration is reserved for later segmentation. This process is done until the whole image is labeled.

\subsection{Iterated Region Merging}
    \begin{figure}[htp]
    \centering
    \subfigure[]
    {\label{IGC:fig:3:grad1}\includegraphics[width=1.7in]{images/IGC/fig3_seeds_initial.pdf}}
    {\subfigure[]
    {\label{fig:3:water1}\includegraphics[width=1.7in]{images/IGC/fig3_1.pdf}}
    \subfigure[]
    {\label{fig:3:grad2}\includegraphics[width=1.7in]{images/IGC/fig3_2.pdf}}
    \subfigure[]
    {\label{fig:3:water2}\includegraphics[width=1.7in]{images/IGC/fig3_4.pdf}}
    \subfigure[]
    {\label{fig:3:water2}\includegraphics[width=1.7in]{images/IGC/fig3_6.pdf}}
    \subfigure[]
    {\label{fig:3:water2}\includegraphics[width=1.7in]{images/IGC/fig3_result.pdf}}}
    \caption{ The iterative segmentation process. (a) Initial segmentation. (b)-(e) show the intermediate segmentation results in the 1st, 2nd, 3rd and 4th iterations. The newly added regions in the sub-graphs are shown in red color and the background regions are in blue color. We can see the target object is well segmented from the background in (f). }
    \label{IGC:fig:iter}
    \end{figure}

 The proposed iterated region merging method starts from the initially segmented image by the modified watershed algorithm in Section \ref{sec:3:algo}. Fig.~\ref{IGC:fig:iter} illustrates the iterative segmentation process by using an example. In each iteration, new regions which are in the neighborhood of newly labeled object and background regions are added into the sub-graph, while the other regions keep their labels unchanged.

 The proposed algorithm is summarized in Table \ref{IGC:Tab:algorithm}. The inputs consist of the initial segmentation from watershed segmentation and user marked seeds. The object and background data models are updated based on the labeled regions from the previous iteration. In Section~\ref{sec:3:Models}, the algorithm to construct data models will be discussed in detail.

    \begin{table}
    \caption{Iterated region merging with localized graph cuts}\label{IGC:Tab:algorithm}
    \begin{tabular}{p{14cm}}
     \hline
      % after \\: \hline or \cline{col1-col2} \cline{col3-col4} ...
    Algorithm1 : RegionMergingGraphCuts()\\
    \textbf{Input}:\\
    -- Initial segmentation of the given image.\\
    -- User labeled object regions $R_o$ and background regions $R_b$.\\
    \textbf{Output}: Segmentation result.\\
     \hline
     1. Build object and background data models based on labeled regions $R_o$ and $R_b$.\\
     2. Build subgraph $G'=<V',E'>$, where $V'$ consists of $R_o$, $R_b$ and their adjacent regions.\\
     3. Update object and background data models using the SelectLabels() algorithm (refer to section
     \ref{sec:3:Models}).\\
     4. Use graph cuts algorithm to solve the min-cut optimization on $G'$, i.e.
     \begin{displaymath}
     \arg\min_{f}U(f|d,f_{N}^{k}).
     \end{displaymath}
     5. Update object regions $R_o$ and background regions $R_b$ according to the labeling results from step 4. \\
     6. Go back to step 2, until no adjacent regions of $R_o$ and $R_b$ can be found. \\
     7. Return the segmentation results.\\
     \hline
   \end{tabular}
   \end{table}

\subsection{Update Object/Background Models} \label{sec:3:Models}
Incorporating user input information in segmentation is one of the most interesting features of graph cuts method \cite{Yuri06}. There is a lot of flexibility in how the information can be used to adjust
the algorithm for a desired segmentation, for example, initializing the algorithm or editing the results. With the given information, the object and background models can be learned for formulating the data term in function~(\ref{IGC:equ:energy1}), which describes how well label $f_p$ fits site $p$. In step 2 of the proposed Algorithm 1 (Table \ref{IGC:Tab:algorithm}), the models are updated based on the previously labeled regions. However, if all the labeled regions are used to update the models,
the misclassified regions will probably reinforce themselves in the
next round of iteration. Therefore, we propose a semi-supervised
approach in which the labeled regions in the $(i-1)^{th}$ iteration are
partially selected to be the seeds for the $i^{th}$ iteration. This
model updating process is independent of the graph cuts optimization
algorithm, aiming to increase the confidence levels of the color
models. The main idea of our object/background models updating
process can be summarized as follows: in each iteration, a set of
confident labels is chosen by a semi-supervised approach, such that
the corresponding regions are taken as confident regions. Based on
these confident regions, new object/background models are
constructed for the graph cuts segmentation, as an integral step of
the proposed Algorithm 1.

There are a number of semi-supervised algorithms which use both labeled and unlabeled data to build classifiers. With the merits of less human effort and higher accuracy, they are of great interest in
practice. The Yarowsky algorithm \cite{Yarowsky} is a well-known semi-supervised algorithm, which is widely used in computational linguistics. Some variants of the original Yarowsky algorithm
\cite{Abney,Haffari} were also developed to optimize specific objective functions. In this section, we adopt it to build better object/background models for the proposed iterated segmentation
algorithm.

   \begin{table}
    \caption{Algorithm of label selection for constructing color model in the $i^{th}$ iteration}\label{IGC:Tab:algorithm_label}
    \begin{tabular}{p{14cm}}
     \hline
      % after \\: \hline or \cline{col1-col2} \cline{col3-col4} ...
    Algorithm2: SelectLabels()\\
    Input: \\
    -- Seeds regions $Y^{0}=R_o \cup R_b$ \\
    -- Labeled regions $X$ after the $1^{st}$ iteration,
    which contain $Y^0$ and their adjacent regions $\bot$.\\
    Output: labeling $Y^{i+1}$. \\
     \hline
     1. For $i\in \{0,1,\dots\}$ do.\\
     2. $\wedge^i=\{x\in X|Y^i\neq \bot\}$.\\
     3. Train classifier on $(\wedge^i,Y^i)$; resulting in
     $\pi^i$.\\
     4. For each example $x\in X$ \\
     \quad 4.1 set $\hat{y}=argmax_j\pi_x^{i+1}(j)$ \\
     \quad 4.2 set \begin{displaymath}
         Y^{i+1} = \left\{ \begin{array}{ll}
         Y_x^0 & \textrm{if $x\in \wedge^0$}\\
         \hat y & \textrm{if $x\in \pi^i \vee\pi_x^{i+1}(\hat{y})>1/L$}\\
         \bot & \textrm{otherwise}
         \end{array} \right.
         \end{displaymath}
      5. If $Y^{i+1}=Y^i$, stop. Otherwise, go to 1. \\
      6. Return $Y^{i}$. \\
      \hline
   \end{tabular}
   \end{table}
Suppose $\phi_x(j)$ is the probability that instance $x$ belongs to the  $j^{th}$ class, and $\pi_x(j)$ is the score of the model in predicting label $j$ for the region $x$. An object function based on
cross-entropy is defined as \cite{Abney}:

\begin{equation}
    l(\phi,\pi)=\sum_{x\in X}H(\phi_x||\pi_x)=\sum_{x\in
    X}\sum_j\phi_x(j)log\frac{1}{\pi_x(j)}\label{equ:yarowskey}
    \end{equation}

    The minimization of function~(\ref{equ:yarowskey}) encourages the unlabeled data becomes labeled, and its assigned label agrees with the model prediction. Since the goal is to build color models based on previously labeled regions, we would like to choose the regions whose predictions are most confident according to the Yarowsky algorithm. With the fact that seeds regions in the $(i-1)^{th}$
    iteration are confident for the graph cut segmentation, we only have to decide which are the confident regions resulting from the graph cut in the $(i-1)^{th}$ iteration. The Algorithm 2 in Table~\ref{IGC:Tab:algorithm_label} describes the process of how to choose the labeled regions in the $(i-1)^{th}$ iteration for constructing color models of the $i^{th}$ iteration, which is corresponding to step 3 in Algorithm 1.
        \begin{figure}[htp]
        \centering
        {\includegraphics[width=4in]{images/IGC/relative_entropy.pdf}}
        \caption{Relative entropy of the object and background distributions (Fig. \ref{IGC:fig:iter}) in different iterations. The three plots represent the red, green and blue color channels respectively. }
        \label{IGC:fig:entropy}
        \end{figure}

     In Algorithm 2, the outer loop is given a seed set $Y^0$ to start with. In step 2, a labeled training set $\wedge^i$ is constructed from the most confident predictions $Y^i$. The score $\pi_x(j)$ is related to all the feature values in a sample $x$ , and is given by:
         \begin{eqnarray}
         \pi_x(j)=\frac{1}{|F_x|}\sum_{f\in F_i}\theta_{fj}
         \end{eqnarray}
         where
         $\theta_{fj}=\frac{|\wedge_{fj}|+1/L|V_f|}{|\wedge_f|+|V_f|}$,
         $|F_x|$ is the number of features of a region $x$, $L$ is the
         number of labels, $|\wedge_{fj}|$ is the number of regions with
         label $j$ and feature $f$; $|V_f|$ and $|\wedge_f|$ are respectively the
         numbers of unlabeled and labeled regions that have feature
         $f$. The feature used here is the average RGB
         color of a region. Abney \cite{Abney} proved that the definition of score $\pi_x(j)$ can promise the object function (\ref{equ:yarowskey}) to decrease with the iteration number until it reaches a minimum. The predicted label for region is given in step 4.1 in Algorithm 2, where it is assumed that the classifier makes confidence-weighted predictions.

To check the relationship between the object and background distributions, we use the relative
entropy to evaluate the distance between them. It is defined as the Kullback-Leibler distance from the distribution of foreground to that of the background, i.e. $D_{KL}(p||q)=\sum_{x\in X}        p(x)log\frac{p(x)}{q(x)}$, where $p(x)$ and $q(x)$ are the probability density functions of the object and background respectively. Fig.\ref{IGC:fig:entropy} shows the value of relative entropy in all 7 iterations for the image in Fig.\ref{IGC:fig:3:grad1}.  As the value of relative entropy goes up from the first iteration, the data models of the object and background become more and more distinguishable. This leads to a higher probability of well separating the object from the background.

 \begin{figure}[htp]
        \centering
        {\includegraphics[width=4in]{images/IGC/energy.pdf}}
        \caption{ The energy evolution of the segmentation results in Fig. \ref{IGC:fig:iter}. Graph cuts energy decreases in the iterated segmentation process.}
        \label{IGC:fig:energy1}
        \end{figure}

  \begin{figure}[htp]
        \centering
        \subfigure[]
        {\label{fig:6:grad1}\includegraphics[width=1.7in]{images/IGC/fig6_iter1.pdf}}
        {\subfigure[]
        {\label{fig:6:water1}\includegraphics[width=1.7in]{images/IGC/fig6_iter2.pdf}}
        \subfigure[]
        {\label{fig:6:grad2}\includegraphics[width=1.7in]{images/IGC/fig6_iter3.pdf}}
        \subfigure[]
        {\label{fig:6:water2}\includegraphics[width=1.7in]{images/IGC/fig6_iter5.pdf}}
        \subfigure[]
        {\label{fig:6:water2}\includegraphics[width=1.7in]{images/IGC/fig6_iter7.pdf}}
        \subfigure[]
        {\label{fig:6:water2}\includegraphics[width=1.7in]{images/IGC/fig6_iter8.pdf}}
        \subfigure[]
        {\label{IGC:fig:6:seeds}\includegraphics[width=1.7in]{images/IGC/fig6_seeds.pdf}}
        \subfigure[]
        {\label{fig:6:water2}\includegraphics[width=1.7in]{images/IGC/fig6_IteratedNL.pdf}}
        \subfigure[]
        {\label{fig:6:water2}\includegraphics[width=1.7in]{images/IGC/fig6_energy.pdf}}
        }
        \caption{Another example of energy evolution. (a)-(f) show the object and
        background seeds in different iterations based on the user input seeds
        shown in (g). (h) shows the final segmentation result, and (i) shows the
         energy values, which are calculated on the whole graphs by using the seeds
         obtained in each iteration. We see that the energy decreases monotonically. }
        \label{IGC:fig:energy2}
        \end{figure}

In the proposed algorithm, segmentation is obtained on different levels of sub-graphs. In light of graph cuts, the segmentation keeps the property of global optimality on each sub-graph. Adding new seeds according to the previous optimal labeling, it increases the amount of useful information that can be used for further segmentation while avoiding introducing much interference information from unknown regions. Fig.\ref{IGC:fig:energy1} shows the energy evolution of the image segmentation process in Fig.\ref{IGC:fig:iter}. Fig.\ref{IGC:fig:energy2} shows another example. With the user input seeds (Fig.\ref{IGC:fig:6:seeds}), the amount of object and background seeds increase automatically based on the segmentation result in each iteration. It is straightforward that our algorithm guarantees the monotonic decrease of energy because iterative minimization can be taken as a multi-step minimization of the total energy.

\section{Experimental Results} \label{IGC:sec:4}
We evaluate the segmentation performance of the proposed method in comparison with the graph cuts algorithm \cite{Yuri01} and $GrabCut$~\cite{grabcut}. Since we use watershed for initial
segmentation, for a fair comparison, we also extend the standard graph cuts to a region based scheme, i.e. we use the regions segmented by watershed, instead of the pixels, as the nodes in the
graph. $GrabCut$ algorithm is also an interactive segmentation technique based on graph cuts and has the advantage of reducing user's interaction under complex background. It allows the user to drag a rectangle around the desired object. Then the color models of the object and background are constructed according to this rectangle. Hence in total we have four algorithms in the experiments: the pixel based graph cuts (denoted by $GC_p$), the region based graph cuts ($GC_r$), the $GrabCut$ and the proposed
iterated region merging method with localized graph cuts (denoted by $IRM$-$LGC$).

In Sections~\ref{IGC:sec:experiment1} and \ref{IGC:experiment:grabcut}, the four algorithms are evaluated qualitatively. In Section \ref{IGC:experiment:QE}, the segmentation results are evaluated quantitatively. Some discussions are made in Section \ref{IGC:experiment:discussion}. Our experiment database contains 50 benchmark test images selected from online resources
\footnote{http://www.research.microsoft.com/vision/cambridge/segmentation/}
\footnote{http://www.cs.berkeley.edu/projects/vision/grouping/segbench/},
 where 10 of them contain objects with simple background and the others are images with relatively complex background. Every image in our database has a figure-ground assignment labeled by human subjects.

\subsection{Comparison with Graph Cuts} \label{IGC:sec:experiment1}
In this subsection, the segmentation results are compared between the proposed algorithm and algorithms $GC_p$ and $GC_r$. Note that $GC_r$ algorithm is used as the first step in lazy snapping \cite{lazy}. This experiment can thus partially compare the performance of lazy snapping and $IRM$-$LGC$. However, a direct comparison of the two methods is not a fair choice, since lazy snapping has another refinement step which adjusts the mis-located boundaries produced by the first step. Fig.\ref{IGC:fig:segment1} shows some images with simple background. In these examples, it is relatively easy to extract the objects from the background. Therefore some of the results by $GC_p$ or $GC_r$ are not too bad, while the proposed method works better.
 \begin{figure}[htp]
        \centering
        {{\label{fig:7:grad1}\includegraphics[width=1.7in]{images/IGC/fig7_bananaseeds.pdf}}
        {\label{fig:7:water1}\includegraphics[width=1.7in]{images/IGC/fig7_musicseeds.pdf}}
        {\label{fig:7:grad2}\includegraphics[width=1.7in]{images/IGC/fig7_stone1seeds.pdf}}
        {\label{fig:7:water2}\includegraphics[width=1.7in]{images/IGC/fig7_bananapixel.pdf}}
        {\label{fig:7:water2}\includegraphics[width=1.7in]{images/IGC/fig7_musicpixel.pdf}}
        {\label{fig:7:water2}\includegraphics[width=1.7in]{images/IGC/fig7_Stonepixel.pdf}}
        {\label{fig:7:seeds}\includegraphics[width=1.7in]{images/IGC/fig7_standard1.pdf}}
        {\label{fig:7:water2}\includegraphics[width=1.7in]{images/IGC/fig7_standard2.pdf}}
        {\label{fig:7:water2}\includegraphics[width=1.7in]{images/IGC/fig7_standard3.pdf}}
        {\label{fig:7:water2}\includegraphics[width=1.7in]{images/IGC/fig7_iter1.pdf}}
        {\label{fig:7:water2}\includegraphics[width=1.7in]{images/IGC/fig7_iter2.pdf}}
        {\label{fig:7:water2}\includegraphics[width=1.7in]{images/IGC/fig7_iter3.pdf}}
        }
        \caption{Segmentation results of images with simple background. The first
        row shows the original images with seeds. Red strokes are for the object
        and the green strokes are for the background.  The second to the forth
        row show the segmentation results by $GC_p$, $GC_r$ and $IRM$-$LGC$ respectively.}
        \label{IGC:fig:segment1}
        \end{figure}

   Extracting objects of interest from complex background is a more
challenging task. Fig.\ref{IGC:fig:segment2} shows some images with
relatively complex background and their segmentation results. In
these images, the objects contain weak boundaries due to poor
contrast and noise, and the colors of some background regions are
very close to those of the objects. Given the same amount of user
input, the proposed $IRM$-$LGC$ achieves much better segmentation
results than the $GC_p$ and $GC_r$ algorithms.
  \begin{figure}[htp]
        \centering
        {{\includegraphics[width=1.7in]{images/IGC/fig8_seeds1.pdf}}
        {\includegraphics[width=1.7in]{images/IGC/fig8_seeds2.pdf}}
        {\includegraphics[width=1.7in]{images/IGC/fig8_seeds3.pdf}}
        {\includegraphics[width=1.7in]{images/IGC/fig8_mashroompixel.pdf}}
        {\includegraphics[width=1.7in]{images/IGC/fig8_boatpixel.pdf}}
        {\includegraphics[width=1.7in]{images/IGC/fig8_fupixel.pdf}}
        {\includegraphics[width=1.7in]{images/IGC/fig8_standard1.pdf}}
        {\includegraphics[width=1.7in]{images/IGC/fig8_standard2.pdf}}
        {\includegraphics[width=1.7in]{images/IGC/fig8_standard3.pdf}}
        {\includegraphics[width=1.7in]{images/IGC/fig8_iter1.pdf}}
        {\includegraphics[width=1.7in]{images/IGC/fig8_iter2.pdf}}
        {\includegraphics[width=1.7in]{images/IGC/fig8_iter3.pdf}}
        }
        \caption{Segmentation results of images with complex background.
        The first row shows the original images with seeds. From the second
        to the forth row, there are the segmentation results obtained by
        $GC_p$ , $GC_r$ and $IRM$-$LGC$ respectively.}
        \label{IGC:fig:segment2}
        \end{figure}

         \begin{figure}[htp]
        \centering
        {{\includegraphics[width=1.7in]{images/IGC/fig9_seeds1.pdf}}
        {\includegraphics[width=1.7in]{images/IGC/fig9_gc1.pdf}}
        {\includegraphics[width=1.7in]{images/IGC/fig9_iter1.pdf}}
        {\includegraphics[width=1.7in]{images/IGC/fig9_seeds2.pdf}}
        {\includegraphics[width=1.7in]{images/IGC/fig9_gc2.pdf}}
        {\includegraphics[width=1.7in]{images/IGC/fig9_iter2.pdf}}
        {\includegraphics[width=1.7in]{images/IGC/fig9_seeds3.pdf}}
        {\includegraphics[width=1.7in]{images/IGC/fig9_gc3.pdf}}
        {\includegraphics[width=1.7in]{images/IGC/fig9_iter3.pdf}}
        {\includegraphics[width=1.7in]{images/IGC/fig9_seeds4.pdf}}
        {\includegraphics[width=1.7in]{images/IGC/fig9_gc4.pdf}}
        {\includegraphics[width=1.7in]{images/IGC/fig9_iter4.pdf}}
        {\includegraphics[width=1.7in]{images/IGC/fig9_seeds5.pdf}}
        {\includegraphics[width=1.7in]{images/IGC/fig9_gc5.pdf}}
        {\includegraphics[width=1.7in]{images/IGC/fig9_iter5.pdf}}
        }
        \caption{Segmentation results by $GrabCut$ and the proposed method. The left column shows the original images with seeds. The blue rectangle is the interaction used in $GrabCut$, while the red and green strokes are the object and background seeds used in the proposed algorithm. The middle column shows the results of $GrabCut$. The right column shows results of $IRM$-$LGC$. }
        \label{IGC:fig:segment3}
        \end{figure}

\subsection{Comparison with GrabCut} \label{IGC:experiment:grabcut}
Fig.\ref{IGC:fig:segment3} compares the results of $IRM$-$LGC$ and $GrabCut$. The left column shows the original images with the seeds points. The middle column shows the segmentation results of $GrabCut$. Implementation of $GrabCut$ uses 5 GMMs to model RGB color data and parameter $\lambda$ is set to be 50. The right column is results of $IRM$-$LGC$. When the objects to be segmented contain similar colors with the background, $GrabCut$ might fail to correctly segment them. Although our algorithm uses more user
interaction than $GrabCut$, this tradeoff leads to more precise segmentation results.

\subsection{Quantitative Evaluation} \label{IGC:experiment:QE}
To better evaluate our algorithm, a quantitative evaluation of the segmentations is given by comparing with ground truth labels in the database. The qualities of segmentation are calculated by using four measures: the true-positive fraction (TPF), false-positive fraction (FPF), true-negative fraction (TNF) and false-negative fraction (FNF):
        \begin{displaymath}
        TPF=\frac{|A_A\cap A_G|}{|A_G|},FPF=\frac{|A_A-A_G|}{|\overline{A_G}|}
        \end{displaymath}
        \begin{displaymath}
        TNF=\frac{|\overline{A_A\cup A_G}|}{|\overline{A_G}|},
        FNF=\frac{|A_G-A_A|}{|A_G|}
        \end{displaymath}
        where $A_G$ represents the area of the ground truth of foreground
        and its complement is $\overline{A_G}$; $A_A$ represents the area of
        segmented foreground by the tested segmentation method.
        Table~\ref{IGC:tab:measures} lists the results of TPF,  FNF, TNF and FPF  by the three methods over the 50 test images. We see the proposed method achieves the best TPF, FNF, TNF and FPF results.

        \begin{table}
         \caption{The TNF, TPF, FNF and FPF results by different methods.}\label{IGC:tab:measures}
        \begin{center}
        \begin{tabular}{|l|c|c|c|c|}
        \hline
        Algorithms & TPF(\%) & FNF(\%) & TNF(\%)& FPF(\%)  \\
        \hline\hline
        $GrabCut$ & 83.65 &16.35&96.59&3.41 \\
        $GC_p$ & 82.72 & 17.28  & 92.37 & 7.63\\
        $GC_r$ &88.01&11.99&93.78&6.22\\
        $IRM$-$LGC$ &91.29&8.71 & 97.75&2.25\\
        \hline
        \end{tabular}
        \end{center}
        \end{table}

As mentioned, the proposed $IRM$-$LGC$ image segmentation method
uses a modified watershed algorithm for initial segmentation. The
median filtering of the gradient image controls the watershed
segmentation output. To examine how the initial segments affect the
final result of $IRM$-$LGC$, we applied the algorithm to different
initial segmentation with different granularities, i.e. different
numbers and sizes of regions in the initial segmentation. This can
be done by changing the filtering times and using different sizes of
filter windows. Fig.\ref{fig:numregions} shows an example. The first
row shows three initial segmentations by the modified watershed
algorithm, where the number of regions is 203, 372 and 1296
respectively. The second row shows the final segmentation results.
We can see that segmentation quality is not sensitive to the initial
segmentation. Fig.\ref{fig:manyregions} compares the segmentation
quality of the same image with 42 different initial segmentations,
from which we can clearly see that the segmentation results are not
 influenced much by the initialization.

 \begin{figure}[htp]
        \centering
        {\includegraphics[width=1.7in]{images/IGC/fig10_203.pdf}}
        {\includegraphics[width=1.7in]{images/IGC/fig10_372.pdf}}
        {\includegraphics[width=1.7in]{images/IGC/fig10_1296.pdf}}
        {\includegraphics[width=1.7in]{images/IGC/fig10_seg203.pdf}}
        {\includegraphics[width=1.7in]{images/IGC/fig10_seg372.pdf}}
        {\includegraphics[width=1.7in]{images/IGC/fig10_seg1296.pdf}}
        \caption{
        Initial segmentation of an image with different numbers of regions.
        In the first row, from the left to the right, there are 203, 372 and
        1296 regions in the initial segmentation respectively. The second row
        shows the final segmentation results.
        }
        \label{fig:numregions}
        \end{figure}

        \begin{figure}[htp]
        \centering
        {\includegraphics[width=3.7in]{images/IGC/fig11_numberregions.pdf}}
        \caption{
        Segmentation qualities vs. initial segmentation in
        different granularities. For the original image used in Fig.\ref{fig:numregions},
        42 different initial segmentations are obtained and used in the proposed
        algorithm. The segmentation quality is measured by TPF, FPF, TNF and FNF scores.
        }
        \label{fig:manyregions}
        \end{figure}

        We use the max-flow algorithm \cite{yuri_pami2} to implement the proposed $IRM$-$LGC$ method.
        The worst case running time complexity for this algorithm is $O(mn^2|C|)$, where $n$ is
        the number of nodes, $m$ is the number of edges and $|C|$ is the cost of the minimum
        cut in the graph. In each iteration of $IRM$-$LGC$, the number of nodes and edges
        are largely reduced in comparison of the pixel based graph cuts algorithm.
        Our experiment is implemented on a PC with Intel Core 2 Duo 2.66 GHz CPU,
        2GB memory. The running time to perform min-cut/max-flow algorithm on the whole
        graph which is based on image pixels is around 10-20ms, while the proposed
        $IRM$-$LGC$ takes far less than 1ms. However, it should be noted that the majority of time for
        our algorithm is spent on constructing color models and updating the graph ($\sim$ 0.3s per iteration), thus the speedup on the min-cut/max-flow part would be relatively modest for the overall algorithm.

    \subsection{Discussion} \label{IGC:experiment:discussion}
    In graph cuts based segmentation, parameter $\lambda$  is used to weight the data and
    smoothness terms. In recent years, some literature~\cite{Kolmogorov_para,Peng2} has studied the
    parameter selection for graph cuts. There are two problems in graph cuts
    algorithm about the selection of $\lambda$. First, given different images, graph
    cuts with a fixed value of $\lambda$ cannot lead to satisfactory segmentation. The
    appropriate $\lambda$ values would vary largely among different images, so the user
    may have to spend a significant amount of time searching for it. Fortunately,
    the proposed $IRM$-$LGC$ is not sensitive to the selection of $\lambda$ across different
    images.  This can be illustrated by the following experiments. In practice we found that
    the region based graph cuts (i.e. $GC_r$) has similar property to pixel based graph
    cuts(i.e. $GC_p$) in parameter selection. Sometimes, $GC_r$ may not lead to satisfying
    segmentation result throughout the searching space of $\lambda$.
    Thus to study on a more general case, the $GC_p$ is used in the following experiments.
    Fig.\ref{fig:lambda1} shows some examples of the segmentation by $GC_p$ and $IRM$-$LGC$. For a
    comparable quality of the segmentation results by the two methods, the best
    value of parameter $\lambda$ in $GC_p$ varies a lot for different images ($2^{nd}$ row in Fig.\ref{fig:lambda1});
     however, a constant $\lambda$ in $IRM$-$LGC$ can lead to satisfying segmentations across
     different images ($3^{rd}$ row in Fig.\ref{fig:lambda1}).

        \begin{figure}[htp]
        \centering
         \subfigure[Images with user input seeds]
        {\label{fig:12:seeds}{\includegraphics[width=1.7in,height=1.2in]{images/IGC/fig11_seeds1.pdf}}
        {\includegraphics[width=1.7in,height=1.2in]{images/IGC/fig11_seeds2.pdf}}
        {\includegraphics[width=1.7in,height=1.2in]{images/IGC/fig11_seeds3.pdf}}}
        \subfigure[$GC_p$,$\lambda=18$]
        {\includegraphics[width=1.7in,height=1.2in]{images/IGC/fig11_ele1.pdf}}
         \subfigure[$GC_p$,$\lambda=50$]
        {\includegraphics[width=1.7in,height=1.2in]{images/IGC/fig11_dog1.pdf}}
         \subfigure[$GC_p$,$\lambda=170$]
        {\includegraphics[width=1.7in,height=1.2in]{images/IGC/fig11_cookie1.pdf}}
         \subfigure[$IRM$-$LGC$,$\lambda=50$]
        {\includegraphics[width=1.7in,height=1.2in]{images/IGC/fig11_ele2.pdf}}
         \subfigure[$IRM$-$LGC$,$\lambda=50$]
        {\includegraphics[width=1.7in,height=1.2in]{images/IGC/fig11_dog2.pdf}}
         \subfigure[$IRM$-$LGC$,$\lambda=50$]
        {\includegraphics[width=1.7in,height=1.2in]{images/IGC/fig11_cookie2.pdf}}
        \caption{
        The values of parameter $\lambda$ in $GC_p$ and $IRM$-$LGC$ for different images.
        }
        \label{fig:lambda1}
        \end{figure}

 The second problem of standard graph cuts is that different values of $\lambda$ will result in very different segmentation results for the same image. Fig.\ref{IGC:fig:lambda2} compares $GC_p$ and $IRM$-$LGC$ by increasing the value of parameter $\lambda$. The original image with user input seeds is in Fig.\ref{fig:12:seeds}. In Fig.\ref{fig:13:pixel1}, $GC_p$ produces a relatively good segmentation with $\lambda=2$. In Fig.\ref{fig:13:pixel2} and Fig.\ref{fig:13:pixel3}, it produces big segmentation errors with $\lambda=50$ and $\lambda=150$, respectively. However, by using $IRM$-$LGC$, we can obtain
similar and good segmentation results for a wide range of values: $\lambda=2$, $\lambda=50$ and $\lambda=150$.

\begin{figure}[htp]
        \centering
        \subfigure[$GC_p$,$\lambda=2$]
        {
        \label{fig:13:pixel1}\includegraphics[width=1.7in,height=1.2in]{images/IGC/fig13_pixel1.pdf}}
         \subfigure[$GC_p$,$\lambda=50$]
        {\label{fig:13:pixel2}\includegraphics[width=1.7in,height=1.2in]{images/IGC/fig13_pixel2.pdf}}
         \subfigure[$GC_p$,$\lambda=150$]
        {\label{fig:13:pixel3}\includegraphics[width=1.7in,height=1.2in]{images/IGC/fig13_pixel3.pdf}}
         \subfigure[$IRM$-$LGC$,$\lambda=2$]
        {\label{fig:13:iter1}\includegraphics[width=1.7in,height=1.2in]{images/IGC/fig11_ele2.pdf}}
         \subfigure[$IRM$-$LGC$,$\lambda=50$]
        {\label{fig:13:iter2}\includegraphics[width=1.7in,height=1.2in]{images/IGC/fig11_ele2.pdf}}
         \subfigure[$IRM$-$LGC$,$\lambda=150$]
        {\label{fig:13:iter3}\includegraphics[width=1.7in,height=1.2in]{images/IGC/fig11_ele2.pdf}}
        \caption{
        Image segmentation with different parameter values. (a-c) show the segmented objects by $GC_p$ and (d-f)
        show the segmented objects by $IRM$-$LGC$. }
        \label{IGC:fig:lambda2}
        \end{figure}

 $IRM$-$LGC$ can reduce greatly the search range of $\lambda$. On most of the test images in our
 database, $\lambda$ is roughly between 50 and 100 for the proposed method, while for $GC_p$,
 the values vary from 10 to 200. An explanation for this is that if the data term in energy function can provide sufficient information for labeling, the graph node does not need a strong relationship with its neighbors. The proposed method gives good object/background models as iteration process goes on, thus the changes of $\lambda$ for various image can be reduced. This brings much benefit for users in real applications.

  Although graph cuts algorithm has relaxed the user input compared with some other algorithms,
  such as livewire \cite{Falaco98}, the input seeds cannot always efficiently indicate the background regions,
  therefore when the connecting regions of the object and background have similar colors, they
  are still hard to be segmented correctly. It is empirically found that if the input seeds
  can cover the main features of the object and background, good segmentation result can be
  obtained.
  Some promising work \cite{R1,R5} has exploited effective methods for arc
weight estimation during the seeds marking process. Their work takes
into account image attributes and object information in order to
enhance the discontinuities between object and background, whereas a
visual feedback can be provided to the user for the next action. We
will investigate how to incorporate these methods into our work in
the future. Fig.\ref{fig:mistake} shows a failure example. The
regions circled in red only connect to object regions on the
sub-graph, so they are easily assigned to the same label. Moreover,
our method uses an initial segmentation to partition the image into
regions, incorrect partition in initialization will also affect the
final segmentation result.

$IRM$-$LGC$ is independent of the initial segmentation step.
However, under-segmented regions from the naive watershed algorithm
cannot be re-partitioned due to the region-merging style of
$IRM$-$LGC$. To reduce the over-segmentation and well keep the
coherence of regions, more sophisticated pre-segmentation algorithms
can be adopted for the initialization. For example, connected
filters with morphological reconstruction operators can eliminate or
merge connected components produced by watershed algorithm
\cite{R3}.  Hence they might be used as a more suitable tool for
improving the initial segmentation quality than median filters.

As in traditional graph cuts algorithm, in the proposed $IRM$-$LGC$
the user input information is also crucial for obtaining desirable
segmentation. Since the newly added seeds in each iteration depend
on the segmentation results in the previous iteration, the
misclassified regions will probably mislead the rest part of
segmentation process. In the future work, other strategies of seeds
selection will be taken into account. For example, the work in
\cite{R1} does not use the seeds from previous delineation to
re-compute the edge weights. It makes the well-segmented regions
unchanged and therefore, the segmentation process becomes more
traceable.

 \begin{figure}[htp]
        \centering
        {\label{fig:14:pixel1}\includegraphics[width=1.9in]{images/IGC/fig14_seeds.pdf}}
        {\label{fig:14:pixel2}\includegraphics[width=1.9in]{images/IGC/fig14_mistakes.pdf}}
        \caption{A failure example of the proposed method.}
        \label{fig:mistake}
        \end{figure}

\section{Conclusion}
This chapter proposed an iterative region merging based image segmentation algorithm by using graph cuts for optimization. The proposed algorithm starts from the user labeled sub-graph and works iteratively to label the surrounding un-segmented regions. It can reduce the interference of unknown background regions far from the labeled regions so that more robust segmentation can be obtained. With the same amount of user input, our algorithm can achieve better segmentation results than the standard graph cuts, especially when extracting the object from complex background. Qualitative and quantitative comparisons with standard graph cuts and GrabCut show the efficiency of the proposed method. Moreover, the search space of parameter $\lambda$ in graph cuts is also reduced greatly by the iterated region merging scheme. 